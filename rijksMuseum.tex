\documentclass[12pt,a4paper]{report}

\begin{document}
  \par
    One of the main goals of the curator of the Rijksmuseum is to make artwork meaningful for the public. This usually takes place in the form of lectures, publications or carefully crafted exhibitions. A visit to the museum should lead to inspiring conversations and discussions about how we can see the world in different and unexpected ways. We believe that leveraging artificial intelligence (AI) to interface with the Rijksmuseum’s Open Access online collection, will contribute significantly to how the people experience the museum’s collection. This also would allow us to visualise the structure underlying artistic creation itself.

    \par
      As a first step in this direction, the Machine Learning team at the Netherlands eScience Center, is working on the Museum-Centered Visual Recognition challenge \cite{Mensink2014}. 

\end{document}
